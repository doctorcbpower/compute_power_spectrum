We construct a cubic mesh of side length $L$ with $N$ mesh cells along each dimension.
Within each mesh cell, we estimate the density fluctuation as
\begin{equation}
  \delta(\vec{x})=\frac{\rho(\vec{x})-\bar{\rho}}{\bar{\rho}}
\end{equation}
where $\bar{\rho}$ is the mean density of the box and $\rho(\vec{x})$ is the density
of the mesh cell at location $\vec{x}$. For a uniform mass resolution cosmological
volume, we can write the mean and local densities as
\begin{equation}
  \bar{\rho}=\frac{M}{L^3}=\frac{N_p\,m_p}{L^3}
\end{equation}
and
\begin{equation}
  \rho(\vec{x})=\frac{N_c\,m_p}{(L/N)^3},
\end{equation}
where $M=\Omega_0\rho_cL^3$ is the mass of the box, $N_p$ is the total number of
particles, $m_p$ is the particle mass, and each mesh cell is of side $L/N$. This means
that we can write the density fluctuation as
\begin{equation}
  \frac{\rho(\vec{x})}{\bar{\rho}}-1=N_c \left(\frac{N^3}{N_p}\right)-1.
\end{equation}



